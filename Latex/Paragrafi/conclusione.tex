\documentclass[../main.tex]{subfiles}

\begin{document}
\section{Counclusione}
Nella teoria dei modelli statistici non parametrici, le B-spline non vengono utilizzate per via del grande incremento di nodi e di basi 
che si vanno a creare utilizzando una griglia Tensore-Prodotto. Hanno quindi ripiegato sulle P-spline. Le HB-spline risolvono in maniera 
più efficiente lo stesso problema, tuttavia si perde la partizione dell'unità. Questa proprietà è importante anche per evitare overfitting 
prematuri, quindi utilizzare le versioni troncate delle HB-spline, chiamate THB-spline, potrebbe risolvere i problemi di overfitting, 
visti negli esempi. Altra cosa da rivedere è il criterio per la rifinitura automatica. Avendo scelto come \textit{funzione di loss} 
MSE, ed essendo 
esso una norma $\mathbf{L^2}$, ciò che minimizza tale norma è la media. Proponiamo quindi di usare i quattro punti con maggiore 
contributo per MSE e usare come range di rifinitura il punto medio tra i primi 2 e il punto medio tra gli ultimi due. Le norme $\mathbf{L^0}$ 
e $\mathbf{L^1}$, minimizzate rispettivamente da moda e mediana, non sono usate in quasi nessun campo applicativo. Viene utilizzata 
in presenza di outliers la norma $\mathbf{L^1}$, ma richiede molti più dati per avere la stessa precisione della norma $\mathbf{L^2}$.
\end{document}