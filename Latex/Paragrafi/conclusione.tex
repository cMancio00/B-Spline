\documentclass[../main.tex]{subfiles}

\begin{document}
\section{Counclusione}
Nella teoria dei modelli statistici non parametrici, le B-spline non vengono utilizzate per via del grande incremento di nodi e di basi 
che si vanno a creare utilizzando una griglia Tensore-Prodotto. Hanno quindi ripiegato sulle P-spline. Le HB-spline risolvono in maniera 
più efficiente lo stesso problema, tuttavia si perde la partizione dell'unità. Questa proprietà è riottenuta utilizzando le versioni troncate delle HB-spline, 
chiamate THB-spline. Altra cosa da rivedere è il criterio per la rifinitura automatica. Avendo scelto come \textit{funzione di loss} 
MSE, le funzioni di rifinitura cercano i due punti che danno maggior incremento alla funzione di loss e raffinano nel range di tali punti. 
Questo procedimento viene iterato fino a che MSE continua a migliorare. Questo può portare ad overfitting se non si introducono criteri d'arresto aggiuntivi.
In aggiunta, si possono verificare grandi discostamenti dalla funzione generatrice, anche avendo un miglior adattamento ai dati.
\end{document}